\ESKDstyle{empty}

\tableofcontents
\thispagestyle{empty} % удаляет нумерацию страниц, которую создает библиотека
\newpage

% = = = = = = = = = = = = = = = =

Цель работы:
научиться разрабатывать и использовать динамические библиотеки (DLL) с использованием Qt.

% = = = = = = = = = = = = = = = =

\section{Уточнение}

Чтобы подключить dll библиотеку, её нужно сбилдить в отдельном проекте, это можно делать как в Visual Studio 2022, 
создав шаблон Dynamic-Link Library (DLL), так и в Qt Creator, выбрав External Library.

Чтобы сбилдить проект в QtCreator выбираем справа сбоку проект,
жмём правой кнопкой мыши и жмём <<Build "...">>.

Проект создаст dll файл в папке <<build-...debug>> / <<Название проекта>> / <<debug>> / <<dll файл>>,
либо в <<build-...release>> / <<Название проекта>> / <<release>> / <<dll файл>>.

Этот dll файл копируем рядом с *.exe файлом графического приложения.

\begin{lstlisting}[numbers=none, name=gpi_copy_dll.bat]
@REM Копируем helper.dll

copy build-gpi_osisp5_lab3-Desktop_Qt_6_2_2_MinGW_64_bit-Debug\gpi_helper\
debug\gpi_helper.dll build-gpi_osisp5_lab3-Desktop_Qt_6_2_2_MinGW_64_bit-Debug\
gpi_osisp5_option5\debug\

copy build-gpi_osisp5_lab3-Desktop_Qt_6_2_2_MinGW_64_bit-Release\
gpi_helper\release\gpi_helper.dll build-gpi_osisp5_lab3-Desktop_Qt_6_2_2_MinGW_64_bit-Release\
gpi_osisp5_option5\release\

@REM Копируем about.dll

copy build-gpi_osisp5_lab3-Desktop_Qt_6_2_2_MinGW_64_bit-Debug\
gpi_about\debug\gpi_about.dll
build-gpi_osisp5_lab3-Desktop_Qt_6_2_2_MinGW_64_bit-Debug\
gpi_osisp5_option5\debug\

copy build-gpi_osisp5_lab3-Desktop_Qt_6_2_2_MinGW_64_bit-Release\
gpi_about\release\gpi_about.dll
build-gpi_osisp5_lab3-Desktop_Qt_6_2_2_MinGW_64_bit-Release\
gpi_osisp5_option5\release\

@REM Копируем helper_class.dll

copy build-gpi_osisp5_lab3-Desktop_Qt_6_2_2_MinGW_64_bit-Debug\
gpi_helper_class\debug\gpi_helper_class.dll
build-gpi_osisp5_lab3-Desktop_Qt_6_2_2_MinGW_64_bit-Debug\
gpi_osisp5_option5\debug\

copy build-gpi_osisp5_lab3-Desktop_Qt_6_2_2_MinGW_64_bit-Release\
gpi_helper_class\release\gpi_helper_class.dll
build-gpi_osisp5_lab3-Desktop_Qt_6_2_2_MinGW_64_bit-Release\
gpi_osisp5_option5\release\
\end{lstlisting}

% = = = = = = = = = = = = = = = =

\section{Задание (хелпер)}

Выбрать 3 вспомогательные функции и вынести их описание и реализацию в динамическую библиотеку helper.dll.
В основном приложении осуществить загрузку реализованных функций во время работы программы
(at run-time, с использованием объекта QLibrary) и их вызов.

\subsection{Проект dll}

\lstinputlisting[language=make]{../gpi_osisp5_lab3/gpi_helper/gpi_helper.pro}

\lstinputlisting[language=C++]{../gpi_osisp5_lab3/gpi_helper/gpi_helper_global.h}

\lstinputlisting[language=C++]{../gpi_osisp5_lab3/gpi_helper/gpi_helper.h}

\lstinputlisting[language=C++]{../gpi_osisp5_lab3/gpi_helper/gpi_helper.cpp}

\subsection{Проект gui}

\lstinputlisting[language=C++]{../gpi_osisp5_lab3/gpi_osisp5_option5/gpi_helper.cpp}

% = = = = = = = = = = = = = = = =

\newpage

\section{Задание (хелпер класс)}

Выбрать вспомогательный класс и вынести его описание и реализацию в динамическую библиотеку helper\_class.dll.
В основном приложении осуществить загрузку реализованного класса во время компиляции (at compile-time).

\subsection{Проект dll}

\lstinputlisting[language=make]{../gpi_osisp5_lab3/gpi_helper_class/gpi_helper_class.pro}

\lstinputlisting[language=C++]{../gpi_osisp5_lab3/gpi_helper_class/gpi_helper_class_global.h}

\lstinputlisting[language=C++]{../gpi_osisp5_lab3/gpi_helper_class/gpi_helper_class.hpp}

\lstinputlisting[language=C++]{../gpi_osisp5_lab3/gpi_helper_class/gpi_helper_class.cpp}

\subsection{Проект gui}

\lstinputlisting[language=C++]{../gpi_osisp5_lab3/gpi_osisp5_option5/gpi_helper_class.hpp}

\lstinputlisting[language=C++]{../gpi_osisp5_lab3/gpi_osisp5_option5/gpi_helper_class.cpp}

% = = = = = = = = = = = = = = = =

\section{Задание (об авторе)}

Реализовать окно «О программе» в виде объекта динамической библиотеки about.dll
с указанием автора программы, группы, курса и краткого описания разработанного приложения.

Осуществить импорт указанной библиотеки и отображение соответствующего окна при выборе пункта меню <<О программе>>.

\subsection{Проект dll}

\lstinputlisting[language=make]{../gpi_osisp5_lab3/gpi_about/gpi_about.pro}

\lstinputlisting[language=C++]{../gpi_osisp5_lab3/gpi_about/gpi_about_global.h}

\lstinputlisting[language=C++]{../gpi_osisp5_lab3/gpi_about/gpi_about.hpp}

\lstinputlisting[language=C++]{../gpi_osisp5_lab3/gpi_about/gpi_about.cpp}

\subsection{Проект gui}

\lstinputlisting[language=C++]{../gpi_osisp5_lab3/gpi_osisp5_option5/gpi_about.cpp}

% = = = = = = = = = = = = = = = =

\section{Задание (изменение шрифта)}

Реализовать расширения для приложения, позволяющие изменять оформление пунктов меню
(шрифт, размер, начертание и т.д.).
Соответствующие изменения должны происходить при выборе специального пункта меню.
Создать как минимум три расширения такого типа.

\subsection{Проект gui}

\lstinputlisting[language=C++]{../gpi_osisp5_lab3/gpi_osisp5_option5/gpi_font.cpp}
